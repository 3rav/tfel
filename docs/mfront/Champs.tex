%%%%%%%%%%%%%%%%%%%%%%%%%%%%%%%%%%%%%%%%%%%%%%%%%%%%%%%%%%%%%%%%%%%%%%%%%%%%%%%
%%	Fichier	   : Champs
%%	Auteur     : th202608@pleiades052.intra.cea.fr
%%	Date       : 13 dc 2010
%%	Rpertoire : /home/th202608/documents/notes/2011/LicosPresentation/
%%%%%%%%%%%%%%%%%%%%%%%%%%%%%%%%%%%%%%%%%%%%%%%%%%%%%%%%%%%%%%%%%%%%%%%%%%%%%%%

% Initialisation des variables par defaut
\centre{CAD}
%% 2005 : \centre{Direction du CEA/Cadarache}
%% mais pas utilise !
\centrenom{Cadarache}
\direction{DEN}
\directionnom{Direction de l'\'Energie Nucl\'eaire}
\departement{DEC}
\departementnom{D\'epartement d'\'Etudes des Combustibles}
\service{SESC}
\servicenom{Service d'\'Etudes et de Simulation du comportement des Combustibles}
\batimentservice{151}
\telephoneservice{23 66}
\faxservice{47 47}

\emetteur{M.~Bauer}

%% ajout pour le modele 2005 :
\directiontitre{Direction de l'\'Energie Nucl\'eaire}
\departementtitre{D\'epartement d'\'Etudes des Combustibles}
\servicetitre{Service d'\'Etudes et de Simulation du comportement des Combustibles}

\laboratoire{LSC}
% \laboratoiretitre{Laboratoire de simulation du comportement des combustibles}
% \batiment{151}
% \telephone{38 62}
% \fax{29 49}
% \mail{sesclsc@drncad.cea.fr}

\approbateur{R.~Masson}


\auteurs{T.~Helfer, V.~Blanc, J.~Julien}
\redacteur{T.~Helfer}
\verificateur{}
\approbateur{R.~Masson}
\emetteur{E.~Touron}

\titre{Le générateur de code \mfront{}}

\date{2013}
% \numero{12-014}
\indice{0}
% \dateversion{09/2012}
\numeroaffaire{A-SICOM-A1-01}
\domaine{DEN/DISN/SIMU}
% \accords{tripartite}
% \clients{AREVA - EDF}
\programmerecherche{SICOM}
\classification{DO}
\motsclefs{
  \mfront{} - \pleiades{}
}

% \codebarre{images/code_barre}
% \diffusionexterne{
% {EDF/R\&D}              & O. Marchand     & 1 & Diffusion par\\
% {EDF/R\&D}              & P. Vasseur      & 1 & courriel     \\
% {EDF/R\&D/MMC}          & P. Ollar         & 1 & \\
% {EDF/R\&D/MMC/CPM}      & N. Prompt       & 1 & \\
%                         & N. Barnel       & 1 & \\
% {EDF/R\&D/MMC/CPM}      & G. Thouvenin    & 1 & \\
%                         & R. Largenton    & 1 & \\
%                         & C. Petry        & 1 & \\
% EDF/SEPTEN              & N. Waeckel      & 1 & \\
%                         & P. Hemmerich    & 1 & \\
%                         & H. Billat       & 1 & \\
%                         & C. Bernaudat    & 1 & \\
% AREVA NP/LA DEFENSE     & L. Catalani     & 1 & \\
%                         & L. Brunel       & 1 & \\
% AREVA NP/LYON           & P. Melin        & 1 & \\
%                         & V. Bessiron     & 1 & \\
%                         & C. Garnier      & 1 & \\                           
%                         & V. Garat        & 1 & \\
%                         & F. Arnoux       & 1 &
% }

\diffusioninterne{
}

% \signatures{-0.}{-39.2}{0.12}{images/signatures.eps}

\stylebib{texmf/bibtex/fr-insa}
\fichierbib{Bibliographie}

%%% Local Variables: 
%%% mode: latex
%%% TeX-master: "utilisation."
%%% End: 
